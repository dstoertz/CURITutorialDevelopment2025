%document class
\documentclass[10pt,oneisde]{book}
%%%% Page Info + Commands %%%%%{

%packages
\usepackage{geometry}
\usepackage{latexsym}
\usepackage{amssymb}
\usepackage{amsfonts}
\usepackage{amstext}
\usepackage{amsmath}
\usepackage{amsthm}
\usepackage{multicol}
\usepackage{hyperref}
\usepackage{enumerate}
\usepackage{tikz}
\usepackage{pgfplots}
\usepackage{xcolor, mdframed}
\usepackage{thmbox}
\usepackage{enumitem}
\usepackage{fancyhdr}
\usepackage{changepage}
\usepackage{xcolor}

\pgfplotsset{compat=1.18}

\renewcommand{\footrulewidth}{0pt}
\setlength{\footskip}{-5mm}

% a good babble textwidth is 5.75in
\newcommand{\babblewidth}{\setlength\textwidth{5.75in}}


% This will stretch out the page
\newcommand{\bigpage}{  \setlength \oddsidemargin{-.25in}
            \setlength \textwidth{6.75in}
            \setlength \topmargin{-1in}
            \setlength \textheight{9.75in}}


%This will shrink the page
\newcommand{\smallpage}{  \setlength \oddsidemargin{.5in}
            \setlength \textwidth{5in}
            \setlength \topmargin{0in}
            \setlength \textheight{9in}}

\newcommand{\separator}{\vglue .1in\hrule\vglue .1in}

\newcommand{\pause}{\vglue .1in\hrulefill {\tiny Pause here}\hrulefill \vglue .1in}

%%general stuff
\newcommand{\caret}{\textasciicircum}

%This will put a circle around something.
\newcommand*\circled[1]{\tikz[baseline=(char.base)]{
            \node[shape=circle,draw,inner sep=2pt] (char) {#1};}}


% Commands for abstract
\newcommand{\Z}{\mathbb{Z}}
\newcommand{\R}{\mathbb{R}}
\newcommand{\C}{\mathbb{C}}
\newcommand{\normal}{\triangleleft}
\newcommand{\Q}{\mathbb{Q}}
\newcommand{\F}{\mathbb{F}}
\newcommand{\N}{\mathbb{N}}
\newcommand{\K}{\mathbb{K}}
\newcommand{\aut}[1]{{\rm Aut}(#1)}
\newcommand{\Ker}{{\rm Ker}\,}
\newcommand{\im}{{\rm Im}\,}
\newcommand{\cyclic}[1]{\langle #1 \rangle}
\newcommand{\isom}{\cong}
\newcommand{\autc}[1]{{\rm Aut_c}(#1)}
\newcommand{\autsub}[2]{{\rm Aut}_{#1}(#2)}

\newcommand{\vp}{\vspace{0.15cm}\\}
\newcommand{\vpp}{\vspace{0.25cm}\\}
\newcommand{\vpn}{\vspace{0.05cm}\\}
\newcommand{\rmv}[1]{\,\backslash\{#1\}}
\newcommand{\rmvs}[1]{\,\backslash{#1}}
\newcommand{\md}[1]{\,\text{mod } #1}

%%%%%%%% command for graphics %%%%%%%%%%%%%
\usepackage{fancyhdr}
%}
\definecolor{darkgreen}{rgb}{0.0, 0.5, 0.0}
\definecolor{sasha}{rgb}{0.0, 0.5, 0.5}
\definecolor{marcus}{rgb}{0.7, 0.3, 0.3}
\definecolor{sam}{rgb}{0.2, 0.2, 0.8}


\begin{document}
\section*{June 4th}
DEF Let $I$ be an ideal in $R$ (commutative ring).
$$\sqrt I = \{r\mid r^n\in I\}$$
\color{red}
EX 
\begin{align*}
    \sqrt{(x^2,y^3)}, \; (x,y)\subseteq \sqrt I\\
    I = (x^2,^3)\\
    \sqrt I \subseteq (x,y)
\end{align*}
\color{blue}
So now let's consider the ideal test. 
$J$ is an ideal of $R$ if:
\begin{enumerate}
    \item [1)]$J$ is closed under addition
    \item [2)]$rJ\subseteq J$, $\forall r \in R$. \\
        Essentially says that it's closed under ANY multiplication.\\
        \color{darkgreen}$\rightarrow $ NOTE: this gives also ADDITIVE INVERSES, so the subgroup test works!.
\end{enumerate}
\color{sasha}
\textbf{Sasha's work}:\vpp
WTS: $rJ\subseteq J,\quad \&(Jr\subseteq R)$ (commutative ring), and the additive subgroup.
\begin{proof}
    Let $r_1,r_2\in J$. We know that $\exists n,m$ such that $r_1^n, r_2^m \in I$. \vp
    Let $a\in R$, WTS $ar_1\in J$.\vp
    $(ar_1)^n=a^nr_1^n\quad\quad r_1^n\in I$\vp
    so $a^nr_1^n \in I_1$ so $(ar_1)^n\in I$.\vp
    Thus, $ar_1\in J$.
\end{proof}
\color{marcus}
\textbf{Marcus work}:\vpp
We want to show that $(r_1+r_2)^{n_1+n_2}\in I\Rightarrow r_1+r_2\in J$.
\begin{proof}
    We have that: 
    $$(r_1+r_2)^{n_1+n_2}=r_1^{n_1+n_2}+b_1r_1^{n_1+n_2-1}+\cdots+b{n_2}r_1^{n_1}r_2^{n_2}+\cdots+r_2^{n_1+n_2}$$
    What's useful is that we can group our terms such that we can pull out a $r_1^{n_1}$ from all of our terms, which we know that $r_1^{n_1}\in I$. \vp
    Furthermore, the second group of terms will at least have a $r_2^{n_2}$ term, and we know that $r_2^{n_2}\in I$. \vp
    So that means every term in the binomial sum is in $I$, and because $I$ is closed under addition, the sum must be in $I$, and therefore the sum must be in $J$. \vp
    Thus, $J$ is an ideal. 
\end{proof}
\color{black}
\newpage
\subsection*{Gandini Notes}{\newcommand{\ux}{\underline x}

We'll start off with some preliminaries. \vpp
DEF Let $A\in GL_n(\K).$\\
We let $A$ "act" on $\underline x=(x_1,\cdots, x_n)$.
by $A\cdot \ux=A\ux \leftarrow \text{MATRIX MULTIPLICATION}$. \vpp\color{red}
DEF (Invariant Polynomial)\\
$f(x_1,\cdots x_n)$ is invariant under $A$\\
if $f(A\ux)=f(\ux)$ (so $f$ stays the same after ACTION)\color{blue}\vpp
DEF Let $G$ be a group of marticies.
If $f(A\ux)=f(\ux), \forall A\in G,$\\
$f$ is an INVARIANT under $G$, on $f\in R^G$
$$R^G=\{f\mid f\text{ is invariant under } G\}$$\vpp\color{sasha}
SUBRING TEST S 
\begin{enumerate}
    \item A constant in $S$
    \item Closed under $+$
    \item Closed under $\cdot$
\end{enumerate}
\color{sam}
\textbf{Sam's Work}:
\begin{enumerate}
    \item $f(\ux)=C$\\
        $\rightarrow f(A\ux)=C$
    \item $f_1,f_@\in R^G$.\\
        WTS $f_1+f_@\in R^G$\\
        $f_2(A\ux)+f_2(A\ux)=f(\ux)+f_2(\ux)\in R^G$\\
    \item $f_1,f_2\in R^G$\\
        WTS $f_2f_2\in R^G$\\
        $f_1(A\ux)f_2(A\ux)$\\
        $=f_1(\ux)f_2(\ux)$ which is in $R^G$. 
\end{enumerate}\color{black}
\subsubsection*{final gandini send-off}
$\K[x^2]$ is the \color{red} subring generated by $x^2$\color{black}\\
$(x^2)$ is the \color{red} ideal generated by $x^2$\\
\color{sasha}$1\in $ SUBRING, \; $1\notin (x^2)$ as $(x^2)\neq R$. \\
\color{darkgreen}$x^3\notin \K[x^2]$, \; $x^2\in (x^2)$
}





\end{document}